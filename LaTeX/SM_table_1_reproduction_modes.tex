\documentclass{article}
\usepackage[a4paper,margin=1in,landscape]{geometry}
\usepackage{cite}

\begin{document}

\thispagestyle{empty}
\begin{table}
  \caption{Cellular mechanisms of asexuality}
  \centering
  % \begin{threeparttable}

    \begin{tabular}{ c c c c c }
      species & common name & reproduction mode & evidence & reference \\
      \hline
      \textit{Poecilia formosa}	&	Amazon molly & sperm-dependent apomixis & Cytology, genetics & \cite{Lamatsch2000} \\
      \textit{Adineta vaga} & bdelloid rotifer & apomixis & & \cite{MarkWelch1998}
    \end{tabular}

  % \begin{tablenotes}
  %   \item[*]  A stange genome.
  %   \item[**] A totally obscure genome.
  % \end{tablenotes}

  % \end{threeparttable}
  \end{table}

  \clearpage
  \newpage

  \bibliography{SM_table_1_reproduction_modes}{}
  \bibliographystyle{plain}
\end{document}

% \documentclass[a4paper,landscape]{article}
% \usepackage{lipsum}
% \usepackage{threeparttable}
% % \usepackage{makecell,booktabs}
%
% \begin{document}
%
%
% \title{SM table 1: Cellular mechanisms of asexuality}
%
%
%
% \lipsum[3-10]
%
% \begin{table}
%   \caption{Cellular mechanisms of asexuality}
%   \centering
%   \begin{threeparttable}
%
%     \begin{tabular}{ c c c c c }
%       species & common name & reproduction mode & evidence & reference \\
%       \hline
%       \textit{Poecilia formosa}	&	Amazon molly & sperm-dependent apomixis & Cytology, genetics & \\
%
%     \end{tabular}
%
% \begin{tablenotes}
%   \item[*]  A stange genome.
%   \item[**] A totally obscure genome.
%   \end{tablenotes}
%
%   \end{threeparttable}
%   \end{table}
% \end{document}